\documentclass[a4paper,10pt]{article}

\usepackage[margin=0.8in]{geometry}
\usepackage[hidelinks]{hyperref}
\usepackage{scrlayer-scrpage}
\usepackage[utf8]{inputenc}

\ihead{May, 2020}
\ohead{random\_mini\_vm ISA}
\chead{\thepage}

\begin{document}
\tableofcontents
\section{Introduction}
I made this out of pure boredom and an active will to procrastinate school work. This is my first attempt at an ISA, don't expect good design, speed, or logical reasons for the way something is implemented. \\\\
All registers are 1 word wide, where a word is 16-bits. There is a single address space which ranges from \texttt{0x0000-0xFFFF}, the \texttt{PC} is set to \texttt{0x0} and the all the other registers contain an undefined value.

\section{Registers}
There are 8 16-bit registers, the most important being the \texttt{PC} program counter and the \texttt{SP} stack pointer. There is also an accumulator \texttt{AC} and index \texttt{ID} register, the accumulator stores the result of arithmetic and the index is used in loops. There are also 4 general purpose registers \texttt{R0-R3}.

\begin{table}[h]
\begin{tabular}{|c|c|}
\hline
\multicolumn{1}{|l|}{\textbf{Register}} & \multicolumn{1}{l|}{\textbf{Location}} \\ \hline
PC & 0x0 \\
SP & 0x1 \\
AC & 0x2 \\
ID & 0x3 \\
R0 & 0x4 \\
R1 & 0x5 \\
R2 & 0x6 \\
R3 & 0x7 \\ \hline
\end{tabular}
\end{table}

% +---------+-------------------------+
% | 0 0 0 0 | 0 0 0 0 0 0 0 0 0 0 0 0 |
% +---------+-------------------------+

% +---------+-------------------------+
% | 0 0 0 0 | 0 0 0 0 0 0 0 0 0 0 0 0 |
% +---------+-------------------------+

% +---------+-------------------------+
% | 0 0 0 0 | 0 0 0 0 0 0 0 0 0 0 0 0 |
% +---------+-------------------------+

\end{document}
